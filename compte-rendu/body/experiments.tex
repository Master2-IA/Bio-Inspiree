%!TEX root = ../Compte-rendu.tex
\section{Caractéristiques communes aux trois expériences}
\subsection{Environnement}
Une grille de taille $50 \times 50$ avec 3 obstacles de taille $5 \times 5$
disposés aléatoirement sur la grille. Un agent (une tortue) positionné aléatoirement
sur cette même grille. Selon les expériences un ou plusieurs débris sont disposés 
aléatoirement sur la grille.
\begin{figure}[H]
	\begin{center}
		\includegraphics[scale=0.3]{diagrams/environnement.png}
		\caption{Exemple d'environnement}
		\label{fig:env}
	\end{center}
\end{figure}
\subsection{Mesure utilisée}
Pour pouvoir comparer les résultats entre les stratégies nous avons défini
la mesure suivante : $$M = \cfrac{\text{Nombre de \emph{ticks}}}{\text{Nombre de débris collectés}}$$
Cette mesure permet de savoir combien, en moyenne, de \emph{ticks} sont nécessaires
pour la collecte d'un débris. Donc plus $M$ est petit plus l'algorithme est performant.
\section{Un seul débris}
%!TEX root = ../../Compte-rendu.tex
%\subsection{Description}
La seconde comparaison sera faite avec cent débris répartis uniformément.
%!TEX root = ../../Compte-rendu.tex

\begin{figure}[H]
	\begin{center}
		\includegraphics[height=8cm]{diagrams/100TrRnd_all.pdf}
		\caption{}
		\label{fig:100Trashes_Rnd}
	\end{center}
\end{figure}

Nous avons pour les différentes stratégies les médianes suivantes :

\begin{figure}[H]
	\begin{center}
		\begin{tabular}{ | c | c | }
			\hline
			\multicolumn{2}{ | c | }{Médianes pour 100 débris distribués uniformément} \\
			\hline
			Brownian & 1468.673 \\
			Equiprobable & 1155.213 \\
			Custom & 974.9208 \\
			Lévy & 911.797 \\
			\hline
		\end{tabular}
	\end{center}
\end{figure}

En partant de la stratégie Brownian et en allant à celle de Lévy
en passant par Equiprobable et Custom, nous pouvons constater
que nous avons de meilleurs résultats, que ce soit pour leur médiane
comme pour leur fiabilité ainsi que leur valeur minimale et maximale.


\section{Cent débris uniformément répartis}
%!TEX root = ../../Compte-rendu.tex
%\subsection{Description}
La seconde comparaison sera faite avec cent débris répartis uniformément.
%!TEX root = ../../Compte-rendu.tex

\begin{figure}[H]
	\begin{center}
		\includegraphics[height=8cm]{diagrams/100TrRnd_all.pdf}
		\caption{}
		\label{fig:100Trashes_Rnd}
	\end{center}
\end{figure}

Nous avons pour les différentes stratégies les médianes suivantes :

\begin{figure}[H]
	\begin{center}
		\begin{tabular}{ | c | c | }
			\hline
			\multicolumn{2}{ | c | }{Médianes pour 100 débris distribués uniformément} \\
			\hline
			Brownian & 1468.673 \\
			Equiprobable & 1155.213 \\
			Custom & 974.9208 \\
			Lévy & 911.797 \\
			\hline
		\end{tabular}
	\end{center}
\end{figure}

En partant de la stratégie Brownian et en allant à celle de Lévy
en passant par Equiprobable et Custom, nous pouvons constater
que nous avons de meilleurs résultats, que ce soit pour leur médiane
comme pour leur fiabilité ainsi que leur valeur minimale et maximale.


\section{Cent débris répartis en dix clusters}
%!TEX root = ../../Compte-rendu.tex
%\subsection{Description}
La seconde comparaison sera faite avec cent débris répartis uniformément.
%!TEX root = ../../Compte-rendu.tex

\begin{figure}[H]
	\begin{center}
		\includegraphics[height=8cm]{diagrams/100TrRnd_all.pdf}
		\caption{}
		\label{fig:100Trashes_Rnd}
	\end{center}
\end{figure}

Nous avons pour les différentes stratégies les médianes suivantes :

\begin{figure}[H]
	\begin{center}
		\begin{tabular}{ | c | c | }
			\hline
			\multicolumn{2}{ | c | }{Médianes pour 100 débris distribués uniformément} \\
			\hline
			Brownian & 1468.673 \\
			Equiprobable & 1155.213 \\
			Custom & 974.9208 \\
			Lévy & 911.797 \\
			\hline
		\end{tabular}
	\end{center}
\end{figure}

En partant de la stratégie Brownian et en allant à celle de Lévy
en passant par Equiprobable et Custom, nous pouvons constater
que nous avons de meilleurs résultats, que ce soit pour leur médiane
comme pour leur fiabilité ainsi que leur valeur minimale et maximale.

%!TEX root = ../Compte-rendu.tex
\section{Les quatre stratégies}
\`{A} chaque étape (\emph{tick}), la tortue peut faire une des trois actions suivantes :
\begin{itemize}
\item (\textbf{A}) Avancer d'une unité
\item (\textbf{D}) Changer aléatoirement de direction
\item (\textbf{S}) S'arrêter et ramasser un débris (\emph{trash}) s'il y en a un
\end{itemize}
Nous avons choisi de comparer les performances de 
quatre stratégies différentes dans trois conditions différentes.

\subsection{Brownian}
La statégie la plus triviale correspond au mouvement brownien où les
trois actions possibles sont faites à la suite en boucle dans cet ordre :
(\textbf{S}), (\textbf{A}) et enfin (\textbf{D}).

\subsection{\'{E}quiprobable}
Une deuxième solution triviale est de tirer une action aléatoire de manière
équiprobable entre (\textbf{A}), (\textbf{D}) et (\textbf{S}).
\subsection{Lévy}
Cette stratégie correspond à la stratégie décrite dans l'article 
"\emph{Lévy flight search patterns of wandering albatrosses}" de 
G.M. Viswanthan, V. Afanasyev, S. V. Buldyrev, E. J. Murphy, 
P. A. Prince et H. E. Stanley. \\
En ce basant sur l'approximation $P(t_i) \sim (t_i + 1)^{-\mu}$ avec $\mu = 2$
(qui défini la probabilité de voler au temps $t_i$) nous
avons défini une fonction qui retourne aléatoirement
un nombre de pas $n \in \llbracket 0; 50 \rrbracket$ à faire en suivant la densité de probabilité de $P$.\\
La stratégie est comme suit :
\begin{enumerate}
\item S'il reste des pas à faire, (\textbf{A})
\item Sinon 
\begin{enumerate}
\item on calcule le nombre de pas $n \in \llbracket 0; 50 \rrbracket$ à faire à partir de la prochaine étape
\item on choisit de manière équiprobable entre (\textbf{D}) et (\textbf{S})
\end{enumerate}
\end{enumerate}

\subsection{Custom}
Cette stratégie vient de l'idée suivante : plus cela fait de temps que l'on s'est arrêté plus on de chance de s'arrêter.
En s'inspirant de la fonction de probabilité $P$ vue précédemment, nous avons
défini la fonction :
$$P_{(\textbf{S})}(t) = 1 - (t + 1)^{-2}$$
Celle-ci définie la probabilité faire l'action (\textbf{S}) où $t$ est le nombre de \emph{ticks} depuis
le dernier (\textbf{S}).\\
Pour chaque étape nous choisissons aléatoirement de faire l'un des actions  
(\textbf{A}), (\textbf{D}) ou (\textbf{S}) avec respectivement une probabilité
de $\cfrac{1 - P_{(\textbf{S})}(t)}{2}$, $\cfrac{1 - P_{(\textbf{S})}(t)}{2}$ et $P_{(\textbf{S})}(t)$. Si (\textbf{S}) est choisi
alors $t$ est remis à zéro, sinon il est incrémenté.
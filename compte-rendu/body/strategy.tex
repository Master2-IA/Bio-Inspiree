%!TEX root = ../Compte-rendu.tex
\section{Les quatre stratégies}
\`{A} chaque étape (\emph{tick}), la tortue peut faire une des trois actions suivantes :
\begin{itemize}
\item Avancer d'une unité
\item Changer aléatoirement de direction
\item S'arrêter et ramasser un débris (\emph{trash}) s'il y en a un
\end{itemize}
Nous avons choisi de comparer les performances de 
quatre stratégies différentes dans trois conditions différentes.

\subsection{Brownian}
La statégie la plus triviale correspond au mouvement brownien où les
trois actions possibles sont faites à la suite en boucle dans cet ordre :
\begin{enumerate}
\item S'arrêter et ramasser un débris s'il y en a un
\item Avancer d'une unité
\item Changer aléatoirement de direction
\end{enumerate}

\subsection{\'{E}quiprobable}
Une deuxième solution triviale est de tirer une action aléatoire de manière
équiprobable.
\subsection{Lévy}
Cette stratégie correspond à la stratégie décrite dans l'article 
"\emph{Lévy flight search patterns of wandering albatrosses}" de 
G.M. Viswanthan, V. Afanasyev, S. V. Buldyrev, E. J. Murphy, 
P. A. Prince et H. E. Stanley. \\
En ce basant sur l'approximation $P(t_i) \sim (t_i + 1)^{-\mu}$
(qui défini la probabilité de voler au temps $t_i$) nous
avons défini une fonction qui retourne aléatoirement
un nombre de pas $n \in \llbracket 0; 50 \rrbracket$ à faire en suivant la densité de probabilité de $P$.

\subsection{Custom}